\documentclass[12pt]{letter}
\usepackage[margin=1in]{geometry}
\usepackage[T1]{fontenc}

\signature{Wei-Lun Cheng, Daniel Wei-Chung Miao, Guang-Di Chang\\
Graduate Institute of Finance\\
National Taiwan University of Science and Technology\\
Taipei 10607, Taiwan\\
\texttt{miao@mail.ntust.edu.tw}}

\address{Graduate Institute of Finance\\
National Taiwan University of Science and Technology\\
Taipei 10607, Taiwan}

\date{\today}

\begin{document}

\begin{letter}{The Editor\\
\textit{Financial Analysts Journal}\\
CFA Institute}

\opening{Dear Editor,}

We are pleased to submit our manuscript, ``When Machines Pass the Test: Professional Certification Signaling Erosion Under AI Disruption,'' for consideration in the \textit{Financial Analysts Journal}.

\textbf{Summary.} This paper develops a Modified Spence Signaling Model to analyze how large language models erode the signaling value of professional certifications such as the CFA designation. Our central result---the Partial Signaling Collapse Theorem---shows that erosion is selective: AI destroys signaling on formalizable abilities (formula recall, algorithmic computation) while preserving it for tacit competencies (ethical judgment, stakeholder reasoning). We calibrate the model empirically using a controlled experiment on 1,032 CFA-style questions across two model generations (GPT-4o-mini and GPT-5-mini).

\textbf{Practitioner relevance.} The paper directly addresses the CFA Institute's strategic challenge: if AI can replicate the cognitive skills the CFA exam measures, does the charter retain its labor market value? We provide an actionable framework---a six-dimensional ability taxonomy with AI replicability indices---that certification bodies can use to identify which competencies to emphasize in future curricula. Our finding that approximately 50\% of CFA curriculum weight is already highly AI-replicable, leaving only $\sim$29\% of pre-AI signaling value, has immediate implications for curriculum reform.

\textbf{Fit with FAJ.} This manuscript aligns with FAJ's mission to bridge academic research and investment practice. The signaling erosion framework is relevant to hiring managers, compliance officers, and continuing education designers across the investment management industry.

This manuscript has not been previously published, is not under consideration at any other journal, and all authors have approved the submission.

\closing{Sincerely,}

\end{letter}
\end{document}
